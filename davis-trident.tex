\documentclass[12pt]{wrceletter}


\name{Dennis Evangelista}
\position{Assistant Professor}
%\email{\href{mailto:evangeli@usna.edu}{\emph{evangeli@usna.edu}}}
\email{evangeli@usna.edu}
\telephone{410-293-6132}

\date{\today}

\usepackage{designature}
\signature{\vspace*{-0.7in}\includesignature\\Dennis Evangelista}
%\signature{Dennis Evangelista\\Assistant Professor} % title not needed if in letterhead
\address{\null} %{105 Maryland Avenue\\Annapolis, MD 21402} % leave blank, provided in letterhead
%\longindentation=0in % to change signature to be flushleft

\begin{document}
\begin{letter}{% recipient address here
Trident Fellowship Selection Committee}

% opening here
\opening{}
\raggedright % if you like this
\setlength{\parindent}{15pt} % if you like this

I am happy to recommend MIDN 2/C Lenny Davis for the Trident Fellowship.  Mr.~Davis is an outstanding midshipman who embodies academic excellence, evidenced by his 3.96 QPR and \#8 OOM, and has high potential for graduate study.  

I was Mr.~Davis’ instructor in ES281C Intro to Drone Technology, the quadrotor version of our ``School of Drones''. In addition, Mr.~Davis is active in the Squad With Autonomous Teammates Challenge (SWAT-C), an ONR-funded effort to examine how autonomy at a squad level can impact Marine Corps operations in a one-block war. As the secretary and drone racing czar for the USNA Student Chapter of the Academy of Model Aeronautics, he has taken an active role in training others in UAS operations. With his experience here at USNA, Mr.~Davis secured an intership at Dam Neck, VA working with NSW on projects relating to UAS. Our goal with School of Drones is to build drone experience early so that midshipmen can reach graduate-level work in autonomy and system design by the time they are upperclass; Mr.~Davis is one of our first midshipmen to fulfill this goal and it has been a delight watching him progress. He is well positioned to be a leader in a future (autonomous) fleet. 

Mr.~Davis intends to pursue graduate education and is well situated to do so.  I believe his early drone experiences will also set him up for outstanding work in graduate school. I am excited to see where Mr.~Davis goes with autonomy, robotics, and controls; and I expect him to drive future transformation wherever he ends up in the fleet.

\closing{~} % provides empty 

%\ps{post script here}
%\encl{enclosure here}
\end{letter}
\end{document}




%MIDN 2/C Lenny Davis
%Trident Application Bio Sheet
%
%GPA: 3.96  
%OOM: 8 of 1,149  
%Major: Honors Weapons, Robotics, and Control Engineering
%Intended Service Selection: Marine Corps Aviation 
%
%Professional And Leadership Experiences 
%Spring semester Brigade Training Sergeant 
%Stamps Scholar
%Plebe Summer Squad Leader
%PROTRAMID Company Commander
%
%Interests and Activities
%Club Water Polo
%Secretary, Academy of Model Aeronautics/MIDN Drone Racing 
%Company Honor Congress Representative
%Battalion Academic Tutor
%
%Talking Points
%With the help of faculty in WRC I was able to secure the Stamps Scholarship. The scholarship is only available to five midshipmen per class. With the scholarship I was able to travel to Atlanta for the Stamps Scholars National Convention.
%Internship at Dam Neck. Over the summer I had the privilege of working with NSW on projects relating to unmanned aerial systems. This was an incredible opportunity to understand the development process and to better understand how our systems are deployed in the field. 
%I have truly found my passion within the Weapons, Robotics and Control engineering major. This being said, I intend to pursue graduate education after my time at the Naval Academy. My dream is to get a master’s degree in robotics or electrical engineering before continuing my naval career. I have worked hard to get 19 credits ahead of my matrix through validation and overloading in an effort to free up time for research as a 1st class midshipman. Furthermore, I have already taken 3 of the 4 required electives for WRC, to include computer vision, introduction to robotics and autonomous vehicles.  
%I took school of drones during its first year as a 3/C midshipman. This class sparked my interest in quadcopters and lead me to pursue UAS in clubs and as a hobby. From there I became involved in the SWAT-C ECA last year and worked on the implementation of unmanned systems in squad level combat. This included working on the quadrotors themselves and training those who were not familiar with drones and flying. 
%
