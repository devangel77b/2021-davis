\documentclass[10pt]{wrceletter}


\name{Dennis Evangelista}
\position{Assistant Professor}
%\email{\href{mailto:evangeli@usna.edu}{\emph{evangeli@usna.edu}}}
\email{evangeli@usna.edu}
\telephone{410-293-6132}

\date{\today}

\usepackage{designature}
\signature{\vspace*{-0.7in}\includesignature\\Dennis Evangelista}
%\signature{Dennis Evangelista\\Assistant Professor} % title not needed if in letterhead
\address{\null} %{105 Maryland Avenue\\Annapolis, MD 21402} % leave blank, provided in letterhead
%\longindentation=0in % to change signature to be flushleft

\begin{document}
\begin{letter}{% recipient address here
University of Southern California\\
Viterbi School of Engineering\\
Department of Computer Science\\
3650 McClintock Avenue\\
Los Angeles, CA 90089}

% opening here
\opening{}
\raggedright % if you like this
\setlength{\parindent}{15pt} % if you like this

I am happy to recommend MIDN 1/C Lenny Davis for the Computer Science (MS) program in Robotics at USC.  Mr.~Davis is an outstanding midshipman who embodies academic excellence and has high potential for graduate study. Davis exudes passion for robotics in and out of the classroom. He was selected as one of five Stamps scholars in the class of 2021 and worked on drone research with the Naval Special Warfare Development Group. Davis' early interest in drone racing has grown to encompass the wider fields of computer vision and autonomous navigation.  

I was Mr.~Davis’ instructor in ES281C Intro to Drone Technology, the quadrotor version of our ``School of Drones''. In addition, Mr.~Davis is active in the Squad With Autonomous Teammates Challenge (SWAT-C), an ONR-funded effort to examine how autonomy at a squad level can impact Marine Corps operations in a one-block war. Davis is president of the Midshipman Drone Racing Team / USNA Student Chapter of the Academy of Model Aeronautics, and has taken an active role in training others in UAS operations, organizing beginner events, simulator nights, and high-impact demos for VIPs.  Our goal with School of Drones is to build drone experience early so that midshipmen can reach graduate-level work in autonomy and system design by the time they are upperclass; Mr.~Davis is one of our first midshipmen to fulfill this goal and it has been a delight watching him progress. He is well positioned to be a leader in a future (autonomous) fleet. 

Regarding Davis' capacity for independent research, I had planned to work with Davis on a project to track eye movements in human drone racing pilots with an aim to train an AI.  Davis is so good, he was selected for the USNA Trident Scholar program and poached by the cyber-physical group Professors Donnal and Kutzer to do visual odometry applied to additive manufacturing closed-loop control and quality assurance. Excellent students like Davis (and my other student, Stites, whom you should also admit) are a rare commodity.

Davis is well situated to pursue graduate education. Moreover, he is able to work with and lead teams.  His academic success and leadership of the Drone Racing Team are balanced with major brigade military leadership responsibilities at the highest levels. Watching Davis with the team, he thoroughly enjoys (and is grounded by) teaching others about quadcopters, helping them out with builds, and attempting to grow the Drone Racing ECA.  I am excited to see where Mr.~Davis goes with autonomy, robotics, and controls; and I expect him to drive future transformation as a future lieutenant of Marines.

\closing{~} % provides empty 

%\ps{post script here}
%\encl{enclosure here}
\end{letter}
\end{document}




%MIDN 2/C Lenny Davis
%Trident Application Bio Sheet
%
%GPA: 3.96  
%OOM: 8 of 1,149  
%Major: Honors Weapons, Robotics, and Control Engineering
%Intended Service Selection: Marine Corps Aviation 
%
%Professional And Leadership Experiences 
%Spring semester Brigade Training Sergeant 
%Stamps Scholar
%Plebe Summer Squad Leader
%PROTRAMID Company Commander
%
%Interests and Activities
%Club Water Polo
%Secretary, Academy of Model Aeronautics/MIDN Drone Racing 
%Company Honor Congress Representative
%Battalion Academic Tutor
%
%Talking Points
%With the help of faculty in WRC I was able to secure the Stamps Scholarship. The scholarship is only available to five midshipmen per class. With the scholarship I was able to travel to Atlanta for the Stamps Scholars National Convention.
%Internship at Dam Neck. Over the summer I had the privilege of working with NSW on projects relating to unmanned aerial systems. This was an incredible opportunity to understand the development process and to better understand how our systems are deployed in the field. 
%I have truly found my passion within the Weapons, Robotics and Control engineering major. This being said, I intend to pursue graduate education after my time at the Naval Academy. My dream is to get a master’s degree in robotics or electrical engineering before continuing my naval career. I have worked hard to get 19 credits ahead of my matrix through validation and overloading in an effort to free up time for research as a 1st class midshipman. Furthermore, I have already taken 3 of the 4 required electives for WRC, to include computer vision, introduction to robotics and autonomous vehicles.  
%I took school of drones during its first year as a 3/C midshipman. This class sparked my interest in quadcopters and lead me to pursue UAS in clubs and as a hobby. From there I became involved in the SWAT-C ECA last year and worked on the implementation of unmanned systems in squad level combat. This included working on the quadrotors themselves and training those who were not familiar with drones and flying. 
%



%
%Talking points: 
%Passion for robotics in and out of the classroom. - School of drones is what really set off my interest in not only drone racing, but pursuing computer vision and autonomous navigation. Was selected as one of 5 Stamps scholars in the class of 21 and given the opportunity to pursue my interest in drones through the NSWDG internship. Getting the ball rolling on the Midshipman drone racing team, organizing beginner events and planning for simulator nights. 
%Capacity for independent research - Selected for the USNA trident scholar program. Working with Professor Donnal and Kutzer for the last 6 months on a visual odometry/additive manufacturing project. Your opinion of me as a student.
%Ability to work with a team/leadership - Was selected for Brigade leadership positions in the hall. I thoroughly enjoy teaching others about quadcopters, helping them out with builds, and attempting to grow the Drone Racing ECA. 
%
%Stanford 		Electrical Engineering	December 1st, 2020
%Carnigie Mellon		Robotics	December 10th, 2020
%Purdue		Mechanical or Electrical/Computer Engineering 	December 15th, 2020
%Johns Hopkins		Robotics	December 15th, 2020
%USC 		Computer Science: Robotics or ?	January 15th, 2021
%University of Pennselvania 		Robotics	February 1st, 2021
